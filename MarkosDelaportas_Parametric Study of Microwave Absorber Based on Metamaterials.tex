\documentclass[12pt]{article}
\usepackage{geometry}
\geometry{ b4paper, total={220mm,320mm}, left=20mm,
            top=15mm, headheight=33pt,includeheadfoot}

\usepackage[parfill]{parskip}  % Activate to begin paragraphs with an empty line rather than an indent

%%%%%%%%%%%%%%%%%%%%%%%%%%%%%%%%%%%%%%%%%%%%%%%%%%%%%%%%%%%%%%%%%%%%%%%
\usepackage[bigfiles]{pdfbase}
\ExplSyntaxOn
\NewDocumentCommand\embedvideo{smm}{
    \group_begin:
    \leavevmode
    \tl_if_exist:cTF{file_\file_mdfive_hash:n{#3}}{
        \tl_set_eq:Nc\video{file_\file_mdfive_hash:n{#3}}
    }{
        \IfFileExists{#3}{}{\GenericError{}{File~`#3'~not~found}{}{}}
        \pbs_pdfobj:nnn{}{fstream}{{}{#3}}
        \pbs_pdfobj:nnn{}{dict}{
        /Type/Filespec/F~(#3)/UF~(#3)
        /EF~<</F~\pbs_pdflastobj:>>
        }
        \tl_set:Nx\video{\pbs_pdflastobj:}
        \tl_gset_eq:cN{file_\file_mdfive_hash:n{#3}}\video
    }
    %
    \pbs_pdfobj:nnn{}{dict}{
        /Type/RichMediaInstance/Subtype/Video
        /Asset~\video
        /Params~<</FlashVars (
        source=#3&
        skin=SkinOverAllNoFullNoCaption.swf&
        skinAutoHide=true&
        skinBackgroundColor=0x5F5F5F&
        skinBackgroundAlpha=0.75
        )>>
    }
    %
    \pbs_pdfobj:nnn{}{dict}{
        /Type/RichMediaConfiguration/Subtype/Video
        /Instances~[\pbs_pdflastobj:]
    }
    %
    \pbs_pdfobj:nnn{}{dict}{
        /Type/RichMediaContent
        /Assets~<<
        /Names~[(#3)~\video]
        >>
        /Configurations~[\pbs_pdflastobj:]
    }
    \tl_set:Nx\rmcontent{\pbs_pdflastobj:}
    %
    \pbs_pdfobj:nnn{}{dict}{
        /Activation~<<
        /Condition/\IfBooleanTF{#1}{PV}{XA}
        /Presentation~<</Style/Embedded>>
        >>
        /Deactivation~<</Condition/PI>>
    }
    %
    \hbox_set:Nn\l_tmpa_box{#2}
    \tl_set:Nx\l_box_wd_tl{\dim_use:N\box_wd:N\l_tmpa_box}
    \tl_set:Nx\l_box_ht_tl{\dim_use:N\box_ht:N\l_tmpa_box}
    \tl_set:Nx\l_box_dp_tl{\dim_use:N\box_dp:N\l_tmpa_box}
    \pbs_pdfxform:nnnnn{1}{1}{}{}{\l_tmpa_box}
    %
    \pbs_pdfannot:nnnn{\l_box_wd_tl}{\l_box_ht_tl}{\l_box_dp_tl}{
        /Subtype/RichMedia
        /BS~<</W~0/S/S>>
        /Contents~(embedded~video~file:#3)
        /NM~(rma:#3)
        /AP~<</N~\pbs_pdflastxform:>>
        /RichMediaSettings~\pbs_pdflastobj:
        /RichMediaContent~\rmcontent
    }
    \phantom{#2}
    \group_end:
}
\ExplSyntaxOff
%%%%%%%%%%%%%%%%%%%%%%%%%%%%%%%%%%%%%%%%%%%%%%%%%%%%%%%%%%%%%%%%%%%%%%%

\usepackage{graphicx}

\usepackage{mathtools}
\usepackage{blindtext}
\usepackage{multicol}
\usepackage{listings, matlab-prettifier}
\usepackage{tikz}
\usepackage{wrapfig}
\usepackage{longtable}
\usepackage{booktabs, caption, colortbl}
\usepackage{pdflscape}
\usepackage{appendix}

\usepackage{pgfplots}
\pgfplotsset{
    compat=1.18,
    label style={font=\tiny\sffamily},
    % legend style={font=\footnotesize},
    % title style={font=\scriptsize}
    no markers,
    every axis plot/.append style={thin},
    every tick label/.append style={font=\tiny\sffamily},
    grid=both,
    tick align=outside,
    tickpos=left,
    grid style={line width=.1pt, draw=gray!10},
    major grid style={line width=.2pt,draw=gray!50},
    minor tick num=3,
    %enlargelimits={abs=0.5},
    axis line style={latex-latex}
}

\usepackage{amssymb}
\usepackage{enumitem}
\usepackage{amsmath}

\usepackage{fancyhdr, lastpage, setspace}
\pagestyle{fancy}
\usepackage{caption, subcaption, zref-totpages}

\usepackage{fontspec}
\usepackage{microtype}

\usepackage{polyglossia}

\usepackage[dvipsnames]{xcolor} %red, green, blue, yellow, cyan, magenta, black, white

%\usepackage{multimedia}

%\usepackage{xmpmulti}

%\def\pdfpageref{\pdffeedback pageref}
\usepackage{hyperref}
\hypersetup{
    bookmarksopen=true,
    colorlinks=true,
    linkcolor=black,
    filecolor=magenta,      
    urlcolor=black,
    pdftitle={Parametric Study of a Microwave Absorver Based on Metamaterials}
}

\setmainlanguage{english}
\setdefaultlanguage{english}
\setotherlanguages{greek}

\usepackage[ backend=biber, bibencoding=utf8, style=ieee]{biblatex}
\addbibresource{ref.bib}

\listfiles

\urlstyle{same}

\rhead{\includegraphics[width=1cm]{ece.png}}
\lhead{\includegraphics[height=1cm]{uowm.png}}

\fancyfoot{}
\fancyfoot[RO]{\thepage /\ztotpages}

\graphicspath{ {img} }
%\addmediapath{ {video} }

\setmainfont[Ligatures=TeX,Numbers=Lining]{Linux Libertine}
\setsansfont[Kerning=On]{Atkinson Hyperlegible}
\setmonofont{Corbel}

\renewcommand{\thesection}{\Roman{section}}
\renewcommand{\thesubsection}{\thesection.\Roman{subsection}}
\renewcommand{\thesubsubsection}{\thesubsection.\Roman{subsubsection}}

\newcommand{\ra}[1]{\renewcommand{\arraystretch}{#1}}

\onehalfspacing

\AddToHook{env/landscape/begin}
 {%
  \clearpage
  \pagestyle{empty}
  \AddToHook{shipout/background}[sven/page]
    {
     \put(0.9\paperwidth,-0.5\paperheight)%adapt values
      {\rotatebox{90}{\thepage}}
    }%
 }     

\AddToHook{env/landscape/after}
 {\RemoveFromHook{shipout/background}[sven/page]}

\title{ \textsf{Parametric Study of a Microwave Absorber Based on Metamaterials.}\\
    %\textsf{Special Project}\\
    \textsf{\Large Department of Electrical \& Computer Engineering}\\
    \textsf{\large University of Western Macedonia}
} \author{\textsf{Markos Delaportas} \footnote{E-mail: ece01316@uowm.gr}}
\date{\textsf{October 2024}}

\begin{document}

\maketitle

\textbf{ \textit{Abstract} -- Microwave absorbers play a crucial role in modern
    telecommunications and electronic systems by mitigating unwanted electromagnetic
    interference (EMI) and enhancing the performance of various devices. These absorbers are
    essential in applications ranging from radar systems and anechoic chambers to consumer
    electronics and medical devices. Traditional microwave absorbers, while effective, often
    suffer from limitations such as bulkiness and narrow bandwidth.Metamaterial-based
    microwave absorbers offer a promising alternative due to their unique electromagnetic
    properties, which are not found in natural materials. These engineered materials can
    achieve near-unity absorption across a wide range of frequencies, making them highly
    efficient. The advantages of metamaterial absorbers include their thin profile,
    lightweight nature, and the ability to tailor their absorption characteristics through
    precise structural design. This makes them ideal for applications requiring compact and
    efficient EMI mitigation. Additionally, metamaterial absorbers can be designed to operate
    over multiple frequency bands, providing versatility and enhanced performance in complex
    electromagnetic environments.}

{\let\clearpage\relax \tableofcontents} \thispagestyle{empty}    
%\tableofcontents


\section{\textsf{Introduction}}
    The study begins with a theoretical exploration of absorber devices and the unique
    properties of metamaterials that make them suitable for electromagnetic wave absorption.
    Following this, the report details the implementation of a specific microwave absorber
    device using advanced simulation software, highlighting the practical aspects of device
    design and performance evaluation. Finally, the report addresses the parametric design and
    optimization of the device, fine-tuning structural parameters to achieve optimal absorption
    characteristics. Through this comprehensive approach, the report aims to provide a thorough
    understanding of the principles, design methodologies, and practical applications of
    metamaterial-based microwave absorbers.

\section{\textsf{Theoretical Study}}
    Metamaterials are artificially engineered materials with unique electromagnetic properties
    not found in nature. They are designed with specific geometrical structures that allow them
    to exhibit properties like negative refractive index, reverse Snell's law, and right/left-handed
    behavior. The first to coin the term of metamaterial absorbers was Victor Veselago \cite{veselago_left_2006}.
    
    An MMA typically comprises three layers: 
    \begin{itemize}
        \item A periodic metallic pattern on top
        \item A dielectric substrate in the middle
        \item A bottom metallic ground plane
    \end{itemize}
    This layered structure enables efficient absorption of electromagnetic waves.

    Impedance matching is crucial for MMAs to minimize reflection and maximize absorption. 
    This is achieved when the impedance of the MMA is matched to the impedance of free space,
    ensuring that incident electromagnetic waves are absorbed rather than reflected.
    
    Reducing the plasma frequency of metals in MMAs allows them to operate effectively at 
    lower frequencies, expanding their applicability to various frequency ranges. 
    This is achieved by manipulating the density of free electron carriers in the metal.

    Multi-layer structures in MMAs enable broadband absorption by creating multiple resonant
    frequencies. By stacking different layers with varying properties, a wider range of 
    frequencies can be absorbed effectively.


    Designing MMAs for specific applications often presents challenges related to achieving 
    the desired bandwidth and absorption ratio. Balancing these requirements while considering
    factors like size, complexity, and cost can be difficult, requiring careful optimization
    of the MMA structure and materials.

    \subsection{\textsf{Impedance Matching Free Space}}
        An absorber can be represented as a transmission line equivalent
        \cite{biswas_ultra-wideband_2022}. In order to maximize the absorbance we need to optimize
        against a specific variable:
        \begin{equation}
            A = 1 - |\Gamma|^2 - |T|^2
            \label{eq:A}
        \end{equation}

    \subsection{\textsf{Magnetic Resonance}}
        Is what can be achieved with closed loop structures \cite{abdulkarim_review_2022}.
    
    \subsection{\textsf{Electrical Resonance}}
        Can be achieved introducing gaps in the structure \cite{abdulkarim_review_2022}.

    \subsection{\textsf{Plasma Frequency}}
        Plasma frequency of a material is the electron cloud oscillations for a specific material.
        \begin{equation}
            \omega_p = \sqrt{\frac{Ne^2}{m\epsilon_0}}
        \end{equation}
    
\section{\textsf{Simulation}}
    The simulation has be created using the CST software in order to implement a configuration such as shown in \cite{zhang_design_2023}.

    \subsection{\textsf{CST Implementation}}
        The vertical layout implemented in CST consists of a three layer structure:
        \begin{itemize}
            \item Dielectric substrate
            \item Air
            \item Metal Backplate
        \end{itemize}

        The dielectric substrate will also embody a metallic component made of the same material as the metal backplate; copper
        (5.96 \mu $10^7$ S/m).

        At first I'll place the substrate without the resonance layer:  
        \begin{figure}[h]
            \centering
            \includegraphics[width=\textwidth]{substrate.png}
            \caption{FR-4 Dielectric Substrate}
            \label{img:substrate}
        \end{figure}

        Later I'll place the metal resonance layer as well but there are a number of possible candidates
        I think of trying to simulate whereas the vertical layout is pretty much fixed.

        I'll place the two other layers below Z=0, turn on the orthographic side
        view to remove shadows and voila: 
        \begin{figure}[h]
            \centering
            \includegraphics[width=\textwidth]{verticaLayout.png}
            \caption{Vertical Layout Orthographic View}
            \label{img:verticaLayout}
        \end{figure}
        I think it really gives a sense of scale as the air layer truly shadows the other two.

        Now its time to add the ring that is of the same material and thickness as the backplate and lies
        on top of the dielectric substrate. 
        \begin{figure}[h]
            \centering
            \includegraphics[width=\textwidth]{ring.png}
            \caption{Ring Resonator}
            \label{img:ring}
        \end{figure}

        Then I will move the substrate, air and backplate layers all below Z=0 just ti make is easier
        with designing the arrows. For this I make the assumption that both the arrow body and point are
        a=0.5mm of width. 
        \begin{figure}[h]
            \centering
            \includegraphics[width=\textwidth]{parallel.png}
            \caption{Initial Arrow Base}
            \label{img:parallel}
        \end{figure}
        In order to accurately place all the curve points that define the arrow some basic calculations
        shall be made. The two points of the arrow base lie exactly on the arc of the ring and are 
        equidistant from curve y=x so the in order to find their cartesian coordinates the following
        system shall be solved. 

        \begin{lstlisting}[frame=single, numbers=left, style=Matlab-Pyglike]
            syms x1 x2

            eq1 = 2*(x1 - x2)^2 == .5^2;
            eq2 = sqrt(x2^2 + x1^2) == 2.7;

            sol = solve([eq1, eq2], [x1 x2]);
            disp([sol.x1 sol.x2]);  
        \end{lstlisting}
        
        \begin{equation}
            \label{eq:xysys}
            \displaystyle \begin{array}{l} 
                \left(\begin{array}{cc} 
                    \sigma_3 -\frac{2916\,\sigma_1 }{1433} & -\sigma_1 \\
                    \sigma_4 -\frac{2916\,\sigma_2 }{1433} & -\sigma_2 \\
                    \frac{2916\,\sigma_1 }{1433}-\sigma_3  & \sigma_1 \\
                    \frac{2916\,\sigma_2 }{1433}-\sigma_4  & \sigma_2  
                \end{array}\right)\\
                \mathrm{}\\
                \textrm{where}\\
                \mathrm{}\\
                \;\;\sigma_1 =\sqrt{\frac{729}{200}-\frac{7\,\sqrt{59}}{80}}\\
                \mathrm{}\\
                \;\;\sigma_2 =\sqrt{\frac{7\,\sqrt{59}}{80}+\frac{729}{200}}\\
                \mathrm{}\\
                \;\;\sigma_3 =\frac{400\,{{\left(\frac{729}{200}-\frac{7\,\sqrt{59}}{80}\right)}}^{3/2} }{1433}\\
                \mathrm{}\\
                \;\;\sigma_4 =\frac{400\,{{\left(\frac{7\,\sqrt{59}}{80}+\frac{729}{200}\right)}}^{3/2} }{1433}
            \end{array}
        \end{equation}

        Which results in two points/quadrant so picking out the two points of the 1st quadrant and 
        inserting them to CST the arrow body is parallel again 
        \begin{figure}[h]
            \centering
            \includegraphics[width=\textwidth]{corretArrowBase.png}
            \caption{Correct Arrow Base}
            \label{img:corretArrowBase}
        \end{figure}

        Then the arrow is mirrored against the X, the Y and the XY planes in order to reach all four 
        sides of the cell, then the face is covered with copper and a height of d=0.035mm is also 
        attributed, which is why it was important to move all other layers below Z=0. 
        \begin{figure}[h]
            \centering
            \includegraphics[width=.4\textwidth]{mirroredArrows.png}\hfil
            \includegraphics[width=.4\textwidth]{RingAndArrows.png}
            \caption{Mirroring Arrow and Cover}
            \label{img:mirrorAndCover}
        \end{figure}
        

        Now I'll try and perform a simulation using the frequency solver in CST just to get an idea 
        how the component behaves, the boundaries will be periodic along the XY plate and I will add
        absorbing conditions along the Z axis.

        For reference the mesh with only the ring element on the surface ends up such as: 
        
        \begin{figure}[h]
            \centering
            \includegraphics[width=\textwidth]{mesh.png}
            \caption{Ring Mesh for reference}
            \label{img:ringMesh}
        \end{figure}

        Now the Mesh that ends up including the arrows is as such:
        \begin{figure}[h]
            \centering
            \includegraphics[width=\textwidth]{mesh.png}
            \caption{Ring Mesh for reference}
            \label{img:RingAndArrowMesh}
        \end{figure}

        Taking a look in the E-Field after the simulation it behaves as such:
        \begin{figure}[h]
            \centering
            \animategraphics[
                width=0.8\textwidth, keepaspectratio, autoplay, type=png,
                controls, loop
            ]{8}{gif/MMA_UnitCell_E_27e2MHz_Zmax1_}{00}{71}
            \caption{Electric Field in 2.7GHz Zmax(1)}
            \label{img:E_2.7_Zmax1}
        \end{figure}
        

    \subsection{\textsf{Alternative Modeling Methods}}
        In order to simulate the absorber there are also a few other ways to go about it:
        \begin{itemize}
            \item Transmission Line equivalent
            \item Electrical circuit equivalent
            \item Mathematical modeling \& code (MATLAB)
        \end{itemize}

        So at first I start by designing a basic layout in CST..
\section{\textsf{Optimization}}
    For this step the IdeM package as well as matlab in order to tweak the parameters that
    describe dimensions\dots

\printbibliography

\end{document}
