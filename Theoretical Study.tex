\section{\textsf{Theoretical Study}}
    Metamaterials are artificially engineered materials with unique electromagnetic properties
    not found in nature. They are designed with specific geometrical structures that allow them
    to exhibit properties like negative refractive index, reverse Snell's law, and right/left-handed
    behavior. The first to coin the term of metamaterial absorbers was Victor Veselago \cite{veselago_left_2006}.
    
    An MMA typically comprises three layers: 
    \begin{itemize}
        \item A periodic metallic pattern on top
        \item A dielectric substrate in the middle
        \item A bottom metallic ground plane
    \end{itemize}
    This layered structure enables efficient absorption of electromagnetic waves.

    Impedance matching is crucial for MMAs to minimize reflection and maximize absorption. 
    This is achieved when the impedance of the MMA is matched to the impedance of free space,
    ensuring that incident electromagnetic waves are absorbed rather than reflected.
    
    Reducing the plasma frequency of metals in MMAs allows them to operate effectively at 
    lower frequencies, expanding their applicability to various frequency ranges. 
    This is achieved by manipulating the density of free electron carriers in the metal.

    Multi-layer structures in MMAs enable broadband absorption by creating multiple resonant
    frequencies. By stacking different layers with varying properties, a wider range of 
    frequencies can be absorbed effectively.


    Designing MMAs for specific applications often presents challenges related to achieving 
    the desired bandwidth and absorption ratio. Balancing these requirements while considering
    factors like size, complexity, and cost can be difficult, requiring careful optimization
    of the MMA structure and materials.