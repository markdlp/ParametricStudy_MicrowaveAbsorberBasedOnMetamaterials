\section{\textsf{Theoretical Study}} Metamaterials are artificially engineered materials
    with unique electromagnetic properties not found in nature. They are designed with
    specific geometrical structures that allow them to exhibit properties like negative
    refractive index, reverse Snell's law, and right/left-handed behavior. The first to
    coin the term of metamaterial absorbers was Victor Veselago \cite{veselago_left_2006}.
    
    An MMA typically comprises three layers: 
    \begin{itemize}
        \item A periodic metallic pattern on top
        \item A dielectric substrate in the middle
        \item A bottom metallic ground plane
    \end{itemize}
    As multi-layer structures in MMAs enable broadband absorption by creating multiple
    resonant frequencies. By stacking different layers with varying properties, a wider
    range of frequencies can be absorbed effectively.

    Impedance matching is crucial for MMAs to minimize reflection and maximize absorption.
    This is achieved when the impedance of the MMA is matched to the impedance of free
    space, ensuring that incident electromagnetic waves are absorbed rather than
    reflected.

    In order to evaluate the absorption of the microwave metamaterial absorber proposed in
    this study \cite{zhang_design_2023} the reflection and transmission power shall be
    calculated as well as the reflection coefficient. The bare necessary equations are
    shown in (\ref{eq:Absorption}).

    \begin{subequations}
        \label{eq:Absorption}
        \begin{align}
            Z & = Z_0 \sqrt{\frac{\mu_r}{\epsilon_r}} \label{eq:Z} \\
            \Gamma & = \frac{Z - Z_0}{Z + Z_0} \label{eq:ReflectionCoeff} \\
            T & = \frac{2Z_0}{Z + Z_0} \label{eq:TransmissionCoeff} \\
            A & = 1 - |\Gamma|^2 - |T|^2 \label{eq:Absorbance}
        \end{align} 
    \end{subequations}

    Simulation tools play a crucial role in the design and analysis of metamaterial
    absorbers. CST Studio Suite is a software package used for high-frequency simulation.
    It offers various solvers, including time-domain and frequency-domain solvers, for
    simulating electromagnetic phenomena \cite{jha_design_2018}.
    