\section{\textsf{Optimization}}
    In this modeling of the absorber the optimization is already done from the reference
    study \cite{zhang_design_2023}. However it is also possible to optimize the electrical
    circuit equivalent using the solvers provided in MATLAB. In greater detail, some
    properties of the absorber can be optimized to better maximize the absorbance,
    such as the \dots and though not other such as \dots

\section{\textsf{Discussion}}
    \par This study has demonstrated the potential of metamaterial-based microwave absorbers
    for achieving efficient electromagnetic wave absorption. The design and simulation of
    a specific absorber device, inspired by the structure presented \cite{zhang_design_2023},
    highlighted the key factors influencing its performance.
    \par The results of the simulation showcase the ability of the metamaterial absorber to
    achieve high absorptivity across a broad range of frequencies. This is attributed to
    the unique electromagnetic properties of metamaterials, which allow for tailoring the
    absorption characteristics through precise structural design. The multi-layered
    structure of the absorber facilitates broadband absorption by creating multiple
    resonant frequencies.
    \par However, it is important to acknowledge the trade-offs associated with optimizing the
    absorber's performance. For instance, while increasing the connection height between
    the copper faces near the arrows might enhance absorbance, it could also introduce
    additional complexities in the fabrication process. Similarly, achieving a perfect
    impedance match between the absorber and free space is crucial for minimizing
    reflection and maximizing absorption, but this can be challenging to achieve in
    practice due to various factors such as material losses and fabrication tolerances.
    \par In terms of future improvements, there are several avenues to explore. One potential
    direction is to investigate different metamaterial structures and configurations to
    further enhance the absorber's bandwidth and absorption efficiency. For example,
    incorporating additional layers or varying the geometry of the metallic patterns could
    lead to improved performance.
    \par Another area of focus could be on reducing the plasma frequency of metals in
    metamaterials. This would allow the absorbers to operate effectively at even lower
    frequencies, expanding their applicability to a wider range of electromagnetic
    environments.  This can be achieved by manipulating the density of free electron
    carriers in the metal.
    \par Furthermore, exploring alternative modeling methods, such as the transmission line
    equivalent and the electrical circuit equivalent, can provide additional insights into
    the absorber's behavior and aid in the optimization process. These methods offer
    different perspectives on the absorber's performance and can be used to validate the
    simulation results obtained from CST.
    \par In conclusion, this study has provided a comprehensive analysis of a
    metamaterial-based microwave absorber, highlighting its potential for efficient EMI
    mitigation. While there are trade-offs to consider and challenges to overcome, the
    future of metamaterial absorbers is promising. By continuing to explore new designs,
    materials, and modeling techniques, it is possible to further enhance their
    performance and expand their applications in various fields.